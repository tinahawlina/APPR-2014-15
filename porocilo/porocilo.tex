\documentclass[10pt]{article}
\usepackage[slovene,english]{babel}
\usepackage{amssymb}
\usepackage{amsmath}
\usepackage[utf8]{inputenc}%omogoča čšćžđ
\usepackage[usenames]{color}
\usepackage{graphicx}
\usepackage{lmodern}
\usepackage[
    paper=a4paper,
    textwidth=15cm,
    textheight=24cm,
    ]{geometry}
\usepackage{fancybox}
\usepackage{url}
\usepackage{epstopdf}
\usepackage{fancyhdr}
\usepackage{eurosym}



\date{\today}
\title{Plavanje-rekordi}
\author{Tina Hawlina}

\begin{document}
\selectlanguage{slovene}
\maketitle

\section{Poročilo}

\subsection{Izbira tematike}

Pri predmetu APPR sem si za temo svojega projekta izbrala plavanje. Izbira teme ni bila preprosta, saj je na internetu na voljo neskončno podatkov, ki bi jih lahko obdelalali. Najprej sem želela izbrati temo globalnega segrevanja ozračja (le za Ameriko), vendar sem ugotovila, da bi bilo verjetno pretežko analizirati toliko podatkov za vsako zvezno državo posebej (temperatura, padavine, snežne padavine, vlažnost, ekstremne vremenske razmere itd.) in jih dobiti v želenih oblikah (.csv, .xsl, .html). Poleg tega so podatki podani po vremenskih postajah, tako da bi bilo potrebno ugotoviti še kje se posamezna postaja nahaja.

Razmišjala sem še o ogromno ostalih temah, npr. delnice, zdravstvo, katerakoli tema is STAT-SI (ker bi bila analiza enostavnješa), naravne nesreče, bonitetne ocene... Med premišljevanjem sem ugotovila, da je zame najpomembnejši vidik enostavnost, saj se s programom R srečujem prvič.

Torej sem končno dobila idejo za temo plavanje. Za to temo sem se odločila predvsem zaradi osebnih preferenc, saj v prostem času kdaj pa kdaj skočim na Fakulteto za šport in odplavam nekaj dolžin, sem pa nekaj časa tudi redno plavala. Predvsem pomembno pa je bilo, da so podatki na voljo v všečnih oblikah in ne preveč zahtevni za prikaz in statistično analizo. 


\subsection{Analiziranje podatkov}

Podatke o doseženih rekordih v plavanju sem pridobila s spletne strani Wikipediain s strani FINA, navajam linka:
$$http://en.wikipedia.org/wiki/List_of_world_records_in_swimming$$
$$http://www.fina.org/H2O/index.php?option=com_content&view=article&id=1271&Itemid=200$$

Ker se pri plavanju tekmuje v različnih slogih, dolžinah, plavamo pa lahko tudi v dveh različnih bazenih-dolžine 50 in 25m, pri čemer daljšemu pravimo olimpijski bazen, kot ste verjetno že vsi slišali. Jaz sem se odločila za analizo rekordov doseženih na dolžinah 50, 100 in 200m, tako na daljšem kot krajšem bazenu v osnovnih štirih tehnikah:
\begin{enumerate}
\item prosto
\item prsno
\item hrbtno
\item delfin
\end{enumerate}

Tako sem si zagotovila, da bo podatkov čimveč, kar je za statistiko pomembno, saj potrebujemo čimvečji vzorec in spet ne preveč, da analiza ne bo pretežavna in da bo računalnik zmogel podatke hitro sprocesirati. Na voljo imam tudi dovolj spremenljivk, vseh treh tipov:
\begin{itemize}
\item urejenostne: proga (50m, 100m, 200m), spol, slogi, dolžina bazena (50m, 25m)
\item številska: čas
\item imenske: državljanstvo postavljalca rekorda, na katerem prvenstvu/olimpijskih igrah je bil rekord postavljen, datum postavitve rekorda
\end{itemize}

\subsection{Cilji naloge}


Glavni cilj je narediti dobro statistično analizo in primerjati podatke tako, da lahko iz njih preberemo karkoli nas zanima in to na uporabniku prijazen način, z grafi, diagrami in podobno, tako da mu ni treba iskati željenega podatka v tabelah. Ostale primerjave po spremenljivkah, ki bi jih rada naredila:
\begin{itemize}
\item med seboj primerjati rekorde v različnih slogih; tako vidimo kateri izmed slogov je najlažji/najtežji
\item ugotoviti, ali so rekordi večinoma postavljeni v olimpijskem bazenu ali bazenu dolžine 25m (pri bazenu 25m na koncu naredimo obrat, pri vsaki tehniki drugačen, pri čemer pridobimo hitrost-odriv od stene)
\item primerjati postavljalce rekorda po državljanstvu, oz. katera država ima največ rekorderjev
\item če so se rekordi tekom časa drastično spremenili(na bolje)
\item če tekmuje več moških ali žensk
\item kje, v kateri državi je bilo postavljenih največ rekordov
\end{itemize}

\end{document}

%\section{Obdelava, uvoz in čiščenje podatkov}

%\section{Analiza in vizualizacija podatkov}

%\includegraphics{../slike/povprecna_druzina.pdf}

%\section{Napredna analiza podatkov}

%\includegraphics{../slike/naselja.pdf}


